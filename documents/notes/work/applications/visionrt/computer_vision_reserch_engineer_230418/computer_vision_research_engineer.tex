% ----------------------------------------------------------------------------------------
% PACKAGES AND OTHER DOCUMENT CONFIGURATIONS
% ----------------------------------------------------------------------------------------

% Font sizes: 10, 11, or 12; paper sizes: a4paper, letterpaper, a5paper, legalpaper, executivepaper or landscape; font families: sans or roman
\documentclass[11pt, a4paper, sans]{moderncv}

% CV theme - options include: 'casual' (default), 'classic', 'oldstyle' and 'banking'
\moderncvstyle{casual}
% CV color - options include: 'blue' (default), 'orange', 'green', 'red', 'purple', 'grey' and 'black'
\moderncvcolor{purple}

% Reduce document margins
% \usepackage[scale=0.6]{geometry}
\usepackage[a4paper, total={410pt, 610pt}]{geometry}
% Uncomment to change the width of the dates column
% \setlength{\hintscolumnwidth}{3cm}
% % For the 'classic'
% \setlength{\makecvtitlenamewidth}{10cm}

% ----------------------------------------------------------------------------------------
% NAME AND CONTACT INFORMATION SECTION
% ----------------------------------------------------------------------------------------

\firstname{Ishan}
\familyname{Costello}

%----------------------------------------------------------------------------------------

\begin{document}
% Print the CV title
\makecvtitle

%-----------------
%   Contact Information
%-----------------

\section{contact information}

\cvitem{\textbf{}email:}{icostello@pm.me \textit{} }
\cvitem{\textbf{}Phone:}{+44(0)7999717524 \textit{} }

% Changes the symbol used for lists
\renewcommand{\listitemsymbol}{}
% \renewcommand{\listitemsymbol}{~~~~}


%----------------------------------------------------------------------------------------
% EDUCATION SECTION
%----------------------------------------------------------------------------------------

\section{education}

\cvitem{2018--present}
{\textbf{PhD: Maximising effectiveness of 3D single-molecule localisation microscopy}}
\cvlistitem{King's College London, London | \textit{underway}}
% \cvlistdoubleitem{CGPA: 7.63}{Highest SGPA: 8.2}

\cvitem{2017--18}{\textbf{MRes Molecular Biophysics} King's College London, London | \textit{Distinction}}

\cvitem{2013--17}{\textbf{BSc Joint Honours Physics \& Philosophy} King's College London, London | \textit{2:1}}

\cvitem{2015--16}{\textbf{Year Abroad} University of British Colombia, Vancouver | \textit{completed}}

\cvitem{2001--2013}{\textbf{European Baccalaureate} European School of Brussels III}

% --------------------------------------
%   SKILLS
% --------------------------------------


\section{technical skills}

\cvitem{programming languages}
{\textbf{image analysis:} highly experienced with \underline{MATLAB} and \underline{Python} (including \underline{os}, \underline{matplotlib}, \underline{tifffile}, and \underline{NumPy} libraries) for image analysis including masking, thresholding, edge detection, convolution, histograms etc.}
\cvlistitem{\textbf{machine learning:} experienced with using \underline{pytorch} and \underline{torchvision} libraries to write advanced discriminative and generative neural networks with deep learning \& convolutional architectures as part of my doctoral research project.}
\cvlistitem{\textbf{markup:} high \underline{{\LaTeX}} competency for drafting various document formats including articles, r\'esum\'es, and my thesis; some competency in \underline{Markdown} for note-taking coupled with \underline{pandoc}.}
\cvlistitem{\textbf{workflow:} touch typist adept with \underline{(neo)vim} in combination with terminal multiplexing in a customised shell in \underline{Linux}-based OS distributions; \textit{venv} virtual environment management in Python; work and document sharing/ project management with \underline{SSH} and \underline{GitHub}}
\cvlistitem{\textbf{currently learning:} \underline{Julia} to future-proof my skillset in machine learning and data processing; \underline{Rust} to gain a deeper understanding of memory management and code performance optimisation.}

\cvitem{languages}
{\textbf{mothertongue} English}
\cvlistitem{\textbf{fluent:} French}
% \cvlistitem{\textbf{basic comprehension:} Dutch, German, Spanish}

\section{relevant experience}

\cvitem{2018--present}{\textbf{PhD Research Project:}
  over the course of my PhD I used Python and ImageJ to conduct quantitative image analysis on large datasets of super-resolved single-molecule localisation microscopy images.
  Additionally, I developed a type of hybrid super-resolution generative adversarial network for achieving resolution isometry in 3D microscope images using PyTorch and torchvision (GitHub link available on request).
  This technology was designed to address the relatively poor axial resolution typical of these types of images, while avoiding classic GAN pitfalls such as structure hallucination.
}

\cvitem{2020--2021}{\textbf{TA: Introduction to Programming for Bioscientists}
  my first experience of Python was being enlisted to teach it to undergraduate students as a teaching assistant.
  This took place during the Covid-19 lockdown and we had to manage the various groups under our supervision on MS Teams.
  My students did well on their exams, and multiple students who were initially unenthusiastic about coding came to enjoy it and continue to pursue it years later.
}

\cvitem{2017--2018}{\textbf{MRes Research Project:}
  development of an algorithm for acquiring and aggregating sarcomeric doublet separation measurements in large datasets of myofibrils.
  I collected large volumes of localisation microscopy images of different sarcomeric epitopes, and developed an image analysis pipeline in MATLAB to aggregate data about the doublet separation, allowing for a high-throughput, high-resolution and information-rich analysis of myofibrillar structures.
}

\end{document}
